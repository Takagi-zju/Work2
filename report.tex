\documentclass{ctexart}

\usepackage{graphicx}
\usepackage{amsmath}
\usepackage{booktabs} 
\usepackage{makecell}

\title{作业二: shell脚本程序设计报告}


\author{邵盛栋 \\ 信息与计算科学 3200103951}

\begin{document}
	
	\maketitle
	\newpage
\section*{shell脚本文件代码}
\begin{verbatim}
	#! /bin/sh
	
	salutation="Hello"
	echo $salutation
	echo "$salutation"
	echo '$salutation'
	echo \$salutation
	echo Hello
	echo "The program $0 is now running"
	echo "The second parameter was $2"
	echo "The first parameter was $1"
	echo "the parameter list was $*"
	echo "The user's home directory is $HOME"
	echo "Please enter a new greeting"
	read salutation
	echo $salutation
	
	
	a="Hello"
	b="hello"
	if [ a = b ]; then
	echo same
	else
	echo different
	fi
	
	
	for foo in hello world !
	do
	echo $foo
	done
	
	
	gender() {
		echo "Are you a man?"
		while true
		do
		echo -n "Enter yes or no:"
		read x
		case "$x" in
		y | yes ) return 0;;
		n | no )  return 1;;
		* )       echo "Answer yes or no"
		esac
		done
	}
	if gender
	then
	echo "Hello,sir."
	else
	echo "Hello,madam."
	fi
	
	exit 0
\end{verbatim}

\section*{输出与说明}
\begin{itemize}
	\item [1)] 
	第一段程序主要学习的是变量的使用,当我需要为变量赋值时,我只需要使用变量名,该变量就会被自动创建。一种检查变量内容的方法是在变量名前加\verb|$|符号,再用\verb|echo|命令将它的内容输出到终端上。如果把一个\verb|$|变量表达式放在双引号中,程序执行到这一行时就会把变量替换为它的值;如果把它放在单引号中,就不会发生替换现象。通过在\verb|$|字符前面加上一个\verb|\|字符以取消它的特殊含义。使用read可以修改变量的值。\\
	对于环境变量和参数变量经过归纳后有如下说明:\\
	\begin{tabular}{cl}
		\toprule
		环境变量 & 说明 \\
		\midrule
		\verb|$HOME| & 当前用户的家目录 \\
		\verb|$PATH| & 以冒号分隔的用来搜索命令的目录列表  \\
		\verb|$PS1| & 命令提示符,通常是\verb|$|字符  \\
		\verb|$PS2| & 二级提示符,用来提示后续的输入,通常是\verb|>|字符  \\
		\verb|$IFS| & \makecell[l]{输入域分隔符。当shell读取输入时,它给出用来分隔单\\词的一组字符,它们通常是空格、制表符和换行符}  \\
		\verb|$0| & shell脚本的名字 \\
		\verb|$#| & 传递给脚本的参数个数 \\
		\verb|$$| & shell脚本的进程号 \\
		\bottomrule
	\end{tabular}\\

	\begin{tabular}{cl}
		\toprule
		参数变量 & 说明 \\
		\midrule
		\verb|$1,$2,……| & 脚本程序的参数 \\
		\verb|$*| & \makecell[l]{在一个变量中列出所有的参数,各个参数之间用环境\\变量IFS中的第一个字符分隔开} \\
		\verb|$@| & \verb|$*|精巧的变体,不使用IFS环境变量 \\
		\bottomrule
	\end{tabular}\\

	赋予第一段程序相应参数,并且输入相应的字符,将得到如下的输出结果:
	\begin{verbatim}
		Hello
		Hello
		$salutation
		$salutation
		Hello
		The program test.sh is now running
		The second parameter was var
		The first parameter was par
		the parameter list was par var
		The user's home directory is /home/student
		Please enter a new greeting
		right
		right
	\end{verbatim}
	
	\item [2)]
	 第二、三两段程序学习的是控制语句的使用,在这里我以if语句和for语句为例。对于这些语句可与其他语言找到共同点,易于理解,包括while语句、case语句、列表等等,其中for语句的循环值通常是字符串。\\
	 运行第两段程序可得到如下结果:
	 \begin{verbatim}
	 	different
	 	hello
	 	world
	 	!
	 \end{verbatim}
	\item [3)]
	第四段程序学习的是shell中的函数功能,我在这里举了一个判断性别的例子。脚本程序开始执行时,函数gender被定义,但先不会执行。在if语句中,脚本程序执行到函数gender时,会根据输入的内容向调用者返回一个值,if结构再根据这个值去执行相应的语句。\\
	输入“yes”后,第四段程序的运行结果如下:
	\begin{verbatim}
		Are you a man?
		Enter yes or no:yes
		Hello,sir.
	\end{verbatim}
\end{itemize}

除上述内容之外,在shell脚本程序中还有两类命令,分别为外部命令(external command)和内部命令(internal command),常用的命令有break命令、:命令、continue命令、.命令、echo命令、eval命令等。在执行命令时,我们可以通过\verb|$(command)|语法来实现获取命令的输出。

\end{document}
